\section{Related Work}\label{sec:signal-related_work}

\noindent
\textbf{Attacks on mobile messaging.}
Mobile messaging services have been hugely popular for decades, but the
SMS protocol was designed primarily for efficiency and not with
privacy in mind \cite{smsbook}. Usability studies have shown that many
users want or even assume that their text messages are private
\cite{HakCha05}, which has made SMS a ``Goldmine to exploit'' for state
surveillance \cite{Ball14,Fif19}. Even encrypted alternatives to SMS are
still targeted by hackers and state-level surveillance tools, as seen
for example by the NSO group's Pegasus spyware, which was used to
target the text messages of journalists and politicians in multiple
countries \cite{MSMRD18}.

\medskip\noindent
\textbf{Statistical disclosure attacks.}
SDAs were first proposed as an attack on mix networks by \cite{SDA}, and
later strengthened to cover more realistic scenarios with fewer or
different assumptions \cite{SDA-MD05,reverse-SDA,two-sided-SDA}.
More recent variants consider the entire network, and attempt to
learn as much as possible about all sender-receiver correlations over a
large number of observations \cite{vida-SDA,LSDA,mix-limits}.
See \cite{meet-sdas} for a nice overview and comparison of many existing
results.

\medskip \noindent
\textbf{Private messaging.}\label{sec:signal-relwork-msging}
Perhaps in response to these highly-publicized attacks, third-party
applications which provide end-to-end encrypted messaging, such as
WhatsApp (since 2016), Telegram, and Signal, are rapidly gaining in
popularity \cite{Lab18}. A good overview for the interested reader
would be the  SoK paper of Unger et.\ al.\ from 2015 \cite{U+15}.

The first cryptographically sound, scalable system for end-to-end
encrypted messaging is the OTR protocol from 2004~\cite{BGB04}, which had
significant influence on the popular systems used today~\cite{Mar13,F+16,C+17}.

%In the meantime, many more advanced cryptographic protocols have been
%investigated in the literature to provide improved functionality,
%security, and privacy \cite{CF10,HLZZ15,CBM15,KLDF16,SVH18,CDAGM19}.
%A good place to begin for the interested reader

% Could cut this if needed?
%\subsection{Anonymous messaging}
Since OTR, significant research has investigated how to remove or hide metadata
to provide anonymous chat applications.  Indeed, similar problems have been noted in mix-nets \cite{leibowitz2019no}. Many such as
Ricochet~\cite{ricochet} rely on Tor~\cite{tor_two}.  Other techniques for obscuring metadata are injecting noise, like
Pond~\cite{pond} and Stadium~\cite{tyagi2017stadium}, or decentralization \cite{kwon2017atom}.  Many of these solutions require
sharing cryptographic identities out-of-band, rather than build off human-friendly
or already known identities.

DC-net based messengers like Dissent~\cite{CF10} or Verdict~\cite{verdict} have also been proposed, but suffer
problems in scaling to the number of users seen on popular messaging
applications~\cite{U+15,HLZZ15}. Others such as
Riposte~\cite{CBM15} have made use of private information
retrieval to achieve anonymity, but this is also expensive in practice.
We focus on sealed sender in this paper, as it is the most widely-deployed
in practice attempt to provide sender anonymity in secure messaging.

% \textbf{Blind Signatures}
% Blind signatures were first introduced by David Chaum in 1982 \cite{C:Chaum82}.  The blind signatures we use in this work are from Chaum and based on RSA.  To blind a message $m$, the client generates a random value $r$ computes $r^e H(m) \mod N$ where $e$ is the server's public exponent.  Signing is computed as $(r^e H(m))^d \mod N$, where $d$ is the server's private key.  The result is $r H(m)^d \mod N$, as $ed=1 \mod N$.  The client can unblind this value by multiplying $r^{-1}$.  The result is a valid RSA signature on $m$.

% Significant research has been devoted to blind signatures since this initial construction. Blind signatures have been proposed based on the discreet log problem \cite{EC:CamPivSta94} and lattices \cite{AC:Ruckert10}.  Versions of blind signatures that support linkage have also been studied \cite{EC:StaPivCam95}.  Most recently, several works have proposed blind signature schemes that support efficient zero-knowledge proofs \cite{SCN:CamLys02, RSA:PoiSan16, RSA:PoiSan18}.
