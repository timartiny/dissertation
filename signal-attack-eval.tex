\section{Attack Evaluation}\label{sec:signal-attack-eval}
In this section, we evaluate our attacks from \cref{sec:signal-attack}
first from a theoretical perspective, and second using a custom simulation.

While our attack is a variant of existing statistical disclosure attacks
(SDAs) in the literature,
the setting is slightly different, and our goals are more modest,
seeking only to de-anonymize the contacts of a single target user. 

\if0
As a
consequence, we can rely on fewer regularity assumptions on other users'
behavior, and discover Bob's contacts with one or two orders of
magnitude fewer observations than comparable SDAs.
\todo{Check 1000s here, or remove this}
This is important for
our setting of anonymous instant messaging, where we might expect only
a few dozen back-and-forth messages in a single conversation, compared
with thousands in a typical network connection.
\fi
\combinedfigures


\subsection{Theoretical analysis of attack success}
\label{sec:signal-attacktheory}

Here we provide statistical bounds to estimate the number of epochs
needed for our attack to successfully de-anonymize one participant in a
conversation. As before, say Bob is the target of the attack, and we
wish to find which other users are communicating with Bob.

Roughly speaking, we demonstrate that (1) all users in conversations with Bob
can be identified provided he is not in too many other simultaneous
conversations with other users, and (2) the number of epochs needed for
this de-anonymization depends logarithmically on the total number of
users. These results hold under some regularity assumptions on
communication patterns, which are most sensible for short periods of
back-and-forth messaging.

\medskip \noindent
\textbf{Statistical Model.} Our statistical analysis relies on the following assumptions:
\begin{enumerate}[nosep]
  \item The probability of receiving a message during any epoch is
    independent of receiving a message during any other epoch.
  \item Each user $u$ (both associates and non-associates) has a fixed
    probability $r_u$ of receiving a message during a random epoch.
  \item Any \emph{associate} $u$ has a fixed probability $t_u$ of
    receiving a message during a target epoch, where $t_u > r_u$.
  \item Every \emph{non-associate} $u$ has the same probability of
    receiving a message during a target or random epoch, i.e., $t_u = r_u$.
\end{enumerate}

The last assumption states that the communications of non-associates is
not correlated with the communication patterns of Bob, which
makes intuitive sense, as they are not involved in conversations with Bob.
% I might remove this next sentence, it seems like a thing reviewers could nitpick,
% but also isn't all that consequential if it's not true.
The regularity (that these probabilities are
fixed and the events are independent)
is most reasonable when considering short attack windows, during
which any user's activity level will be relatively constant.

\medskip \noindent
\textbf{Theoretical attack success bound.} In our attack, all users in the system are ranked according
to their chances of being an associate of Bob after some number of
target and random epochs. We now provide bounds on the number of epochs
necessary to ensure that an arbitrary associate Alice is ranked higher
than \emph{all} non-associates.

\begin{theorem}\label{thm:stats}
  Assume $m$ total users in a messaging system.
  Let Alice be an associate of the target Bob with
  probabilities $r_a, t_a$ of appearing in a random or target epoch
  respectively. Then, under the stated assumptions above, the
  probability that Alice is ranked higher than all non-associates after
  $n$ random and target epochs is at least
  \[1 - \frac{m}{c_a^{\phantom{a}n}},\]
  where $c_a = \exp((t_a-r_a)^2/4) > 1$ is a parameter that depends only
  on Alice's probabilities $t_a$ and $r_a$.
\end{theorem}

The proof is a relatively standard analysis based on Hoeffding's
inequality \cite{hoeffding}, and can be found in \cref{sec:signal-stats-proof}.

We point out a few consequences of this theorem:
\begin{itemize}[nosep]
  \item The success of the attack depends only on the target user Bob
    and his sought-after associate Alice, not on the relative activity
    of any other users.
  \item The number of epochs needed to de-anonymize Alice with high
    probability scales \emph{logarithmically} with the total number of
    users.
  \item The attack succeeds most quickly when Bob is in few other conversations
    (so $t_a$ is large) and
    Alice is communicating mostly just with Bob (so $r_a$ is small).
\end{itemize}

The following corollary, which results from solving the inequality of
\cref{thm:stats} and applying a straightforward union bound, gives an
estimate on how many epochs are necessary to discover all of Bob's
contacts with high probability.

\begin{cor}\label{cor:epochs}
  Let $0<p<1$ be a desired probability bound, and
  assume $m$ total users in a messaging system, of whom $b$ are
  associates of a target user Bob, where the $i$'th associate has
  probabilities $r_i,t_i$ of appearing in a random or target epoch
  respectively. Then, under the previous stated assumptions, with
  probability at least $p$, all $b$ associates of Bob are correctly
  identified after observing
  \[\frac{4}{\min_i (t_i-r_i)^2}\left(
    \ln(m) + \ln(b) + \ln\left(\tfrac{1}{1-p}\right)
  \right)\]
  target and random epochs.
\end{cor}

Comparing to prior work, the closest SDA which has a similar theoretical
bound is from Danezis~\cite{SDA}%
\footnote{Unfortunately, there appear to be at least three slightly different
versions of this bound in the published literature
(\cite[equation (6)]{SDA}; \cite[equation (9.8)]{danezis-thesis};
\cite[page 5]{SDA-MD05}), making it difficult to compare bounds.}.
That work makes much stronger regularity assumptions than our model,
assuming essentially that (1) all epochs contain the same number of
messages (2) every target epoch contains exactly one reply from Bob, (3)
Bob receives a message from each associate with uniform probability, and
(4) all other users, and recipients, are selected uniformly at random
from all $m$ users. Later work also includes a
theoretical bound~\cite{LSDA}, but their model is much more general than ours, where
they seek to reveal the entire network rather than a single target
user.

% \dan{note, we don't look good in the following comparison}
% In our setting, these additional restrictions from \cite{SDA} mean that
% each associate $i$ has the same target and random epoch probabilities,
% $t_i=t_a=\frac{1}{b}$ and $r_i = r_a$, and the mix threshold (for their
% setting) is exactly $mr_a$.
% Then, in our context, the minimum number of rounds of
% observation required by \cite{SDA}\cref{thesisnote}, in order to
% identify all associates with 99\% probability, is
% \[9b^2\left(\sqrt{\frac{m-1}{m^2}(mr_a - 1)} +
% \sqrt{\frac{m-1}{m^2}(mr_a - 1) + \frac{b-1}{b^2}}\right)^2.\]



\subsection{Attack simulation}
\label{sec:signal-attacksim}
%\subsection{Simulations} % description of the simulations, different graph
% structures, include plots of multiple graph structures, comparison to python
% list model, effect of longer epochs

%\TODO{needs a pass for attack 3}


We cannot directly validate the effectivenesses of our attacks in practice, as
we do not have access to Signal's servers and there is no public sample dataset
of Signal sealed sender messages. Instead, we perform \emph{simulations} based
on generalized but realistic assumptions on message patterns. We do not claim
our simulations will reveal the exact number of messages needed to deanonymize
a particular user, as that would depend on exact messaging patterns.
Rather, our simulations give a sense of the order of magnitude of messages
needed to deanonymize a user.


We simulated sequences of target and random epochs (e.g. epochs where Bob does
or does not receive a message) and ranked users by their score. Recall that
a user's score increases if they appear in a target epoch. We simulated
1~million active users, with 800 messages per epoch. This corresponds to users
sending on average about 70 messages per day, with 1~second epochs\footnote{Based off our observation of round-trip delivery receipt times}.

Within each epoch, we select a random set of 800 message destinations. In a
target epoch, Alice (the associate) is sent a message to represent Bob's
delivery receipt that would be sent to her automatically. The remaining messages
are chosen randomly: 25\% of messages are selected as ``repeat'' messages (same
sender and receiver) from
prior epochs (representing one side of a prior conversation), and another 25\%
are selected as ``responses'' to messages in prior epochs (representing a
conversation's response). The remaining 50\% of messages are messages from and
to a random pairing of users from the set of 1~million active users. We find
that the percent of repeats/replies has limited impact on the number of epochs
to identify an associate until over nearly all messages are repeats (i.e. each
epoch is essentially the same small set of senders/receivers). We choose half of
the epochs to be target epochs (where Alice messages Bob) and half as random
(where Alice does not message Bob).


\medskip \noindent
\textbf{Social graph significance.}
We note our experiment does not rely on a particular social graph (or rather,
assumes a fully connected one), as any user
can message any other. In preliminary experiments, we examined the impact 
of several different graph
generators that are designed to simulate social networks, but found 
no noticable change in our results. Specifically, we used
the Erd{\"o}s-R{\'e}nyi~\cite{erdos1959random} model,
Barab{\'a}si-Albert~\cite{albert2002statistical} model,
Watts-Strogatz~\cite{watts1998collective} model, and a fully connected graph,
but found they all resulted in a similar number of epochs
needed to deanonymize the associate (Alice). Given this result, we opted to use
the fully connected graph model for simplicity.




% \VariantThree
\cref{fig:signal-combinedfigure} shows the result of several attack simulations.
We ran each attack simulation for 100 runs, and at each epoch, report the
average rank of Alice's score based on our attack.
% Alice only
First, in the ``Alice Only'' variant, only Alice messages Bob (and no one else).
Even though there are thousands of other users messaging randomly, Alice's score
quickly becomes the top ranked user: within 5 messages, she is uniquely
identified as messaging Bob. 

% Multiple users
If multiple users are also messaging Bob while Alice
does, it takes more total epochs to identify Alice (and her co-associates
messaging Bob). In this scenario, each target epoch is selected to be either
Alice or one of 5 co-associates that messages Bob (6 total conversations with Bob).

% Popular users
If there are popular users present (e.g. users that receive messages in a large
fraction of all epochs), then it may be more difficult to identify Alice without
accounting for them. However, since we remove users that also appear in a large
fraction of random epochs, Alice is still eventually ranked uniquely as messaging Bob.

% We simulated 1000 such popular users, and show that Alice's
% average rank does not fall below this when we do not account for them (``Popular
% users variant 2'' line). 


% combined
Finally, we combine our popular users and multiple messagers into a single simulation,
which is dominated by the multiple messagers effects.

\medskip \noindent
\textbf{Summary.}
In the worst case, it takes on the order of 60 epochs to identify the users
messaging Bob. Note that only half of these are messages to Bob, and the other half are random epochs.
If only one person is messaging Bob, the number of messages needed is under 5 to
identify Alice as the associate of Bob.


% Move to discussion/solution section?
% \EpochEffect


\if0
The Erd{\"o}s-R{\'e}nyi model
assumes that each pair of nodes is equally likely to be connected, and connects
nodes accordingly, while the Barab{\'a}si-Albert model attempts to mimic systems
with certain nodes being more popular (and more connected) than others, and the
Watts-Strogatz model attempts to form clusters, or cliques, in the generated
graph
\fi



\if0
%%%%%%%%%%% OLD STUFF:

%As described above, there are multiple options users could have enabled that
%would impact the attack.  Most importantly, if users have Typing Notifications
%enabled, their devices would generate messages that might pollute the data
%collection process. We start our analysis with the simplifying assumption that

%We start by evaluating Attack 1, which assumes
%both users have typing notifications disabled, the default setting, and that
%only Alice is messaging Bob.
%Additionally, we assume that Bob's device ``immediately'' responds with an
%easily identifiable \textit{delivery receipt}. We remove these simplifying
%assumptions in Attack 2, described in Section~\ref{sssec:attack}

To determine the effectiveness of our attacks, we generate simulated
communication logs that can be processed by the attacker. These logs represent
the information an adversarial service provider (e.g. Signal) could observe; in
the case of sealed sender messages, only the destination and time of messages
are included in the log.

To generate realistic communication logs, we start by generating a
{\em graph} of simulated users, with nodes as users and edges signifying a
messaging relationship between users. Our graph simulation and evaluation code
is 500 lines of C code, using the igraph~\cite{igraph} library to generate
graphs. We discuss our graph generation methodology in more detail next in
\cref{sssec:signal-graphgeneration}.
%\adam{Should we maybe consider the likelihood two friends will communicate? Like as a label on an edge? }

Communication logs are then generated based on the graph.  Because the attack is
only concerned about periods of time in which a message to Bob is observed, we
limit our log generation to those time periods.  During each of these epochs, a
random subset of connected nodes {\em also} communicate, adding messages into
the communication log. We discuss communication log generation in more detail in
\cref{sssec:signal-loggeneration}.

Finally, we use the generated logs to launch the attacks. For each epoch the
list of users that received a message are saved and intersected or tallied with
the list from the other epochs. We discuss our attack evaluation in
\cref{sssec:signal-attack}.

%This is continued until a single sender is identified
%as part of the communicating pair, Alice and Bob for some target, Bob. More
%details, and variants of this attack are discussed in \cref{sssec:attack}.


We performed a Monte-Carlo simulation based on our generated logs for both
attacks to determine how quickly an adversary could identify that Alice was
sending messages to Bob in the presence of other traffic.
As summarized in \cref{fig:signal-NumMessages}, with a user base of 1~million
users, our simulation of Attack~1 finds that on average, Alice can be identified
(deanonymized) after she sends 6.5~messages to Bob.
%We now
%examine graph generation, log generation and the attack in more detail.

\FigIdentifySender

\subsubsection{Graph Generation}
\label{sssec:signal-graphgeneration}

Without access to a real social messaging graph, we are limited to using graph
structures in the literature. 
We examined multiple graph generators that aim to
simulate social networks, including the
Erd{\"o}s-R{\'e}nyi~\cite{erdos1959random} model,
Barab{\'a}si-Albert~\cite{albert2002statistical} model, and the
Watts-Strogatz~\cite{watts1998collective} model. The Erd{\"o}s-R{\'e}nyi model
assumes that each pair of nodes is equally likely to be connected, and connects
nodes accordingly, while the Barab{\'a}si-Albert model attempts to mimic systems
with certain nodes being more popular (and more connected) than others, and the
Watts-Strogatz model attempts to form clusters, or cliques, in the generated
graphs. As we show in \cref{fig:signal-NumMessages}, the structure of the graph
does not significantly impact the effectiveness of the attack. Indeed, we even
simulate the attack on a complete graph, in which any user might communicate
with any other user. This shows that the structure of the graph does not impact
the attack.

\subsubsection{Log Generation}
\label{sssec:signal-loggeneration}

Once the graph structure has been fixed, we use it to generate a communication
log. Because the attacker need only look at messages that were sent around the
same time Bob received a message, we only generate the portions of the overall
communication log that are near those events.  We fix the messaging rate for
Signal, assuming 1 million daily users at 579 messages per second. This number
assumes that each Signal user would send approximately 50 messages each day.

We then select the length of time after a message to Bob that the adversary
watches for potential response (receipt) messages. We refer to this length of
time as an \emph{epoch}, and by default choose a length of 1 second. This is
natural for delivery receipts which are sent immediately if the receiving device
(Bob) is online. Given our messaging rate, we then generate the simulated
messages that would be sent during that epoch. We also consider epochs of
different lengths; the effects varying the epoch length can be seen in
\cref{fig:signal-epoch-effect}.


Naturally, a limitation of our analysis is that it misses message rate
variation that may be present in a true distribution of messages due to
population timezeones or message bursts around particular events or times.
Nonetheless, we believe our simplified model still captures the most important
dynamics relevant to our attack.

\EpochEffect

Each epoch is filled with a fixed number of messages based off conservative
estimations of Signal usage, assuming that each user will send 50 Signal sealed
sender messages each day. From this the number of messages in each epoch can be
calculated from the number of users in the system.

As each epoch represents a message being sent to Bob, the message in the
following epoch are generated as follows: First, for Attack~1, a message to
Alice always appears in the following epoch, representing the delivery receipt
from Bob for her message. For the first epoch, the other messages are generated
by selecting a random node as a sender and choosing one of their possible
recipients at random.

\PercentageEffect


In subsequent epochs, messages are generated to include ``responses'' for
messages sent in previous epochs. 50\% of the messages will still be chosen {\em
randomly}, unrelated to previous epochs (again selected by choosing random users
and selecting random connected peers as the message destination). 25\% of the
messages are {\em replies} to messages sent in any previous epochs, switching
the `to' and `from'\footnote{Although this attack and Signal do not have access
to the `from' fields of Sealed Sender messages, we use the identity of the
sender for the purpose of log generation only we include the `from' field.}
fields of a previous message (e.g. if Charlie sent Dave a message, a reply would
have Dave sending Charlie a message in a subsequent epoch).  Finally, 25\% are
{\em repeated} messages from a previous epoch (the same `to' and `from' fields
as a previous message).
%TODO: what happens when we choose different % for each? is there a significant
%impact on our attack based on changing these parameters?

For Attack~2, where not all messages generate replies to Alice, we only include
Alice in an epoch with probability $f$.  In the remaining $(1-f)$ fraction of
the epochs, Alice's message is replaced with one from a random user.

% below is not exactly true, needs to be cleared up
% maybe just say that we'll only be looking at complete model (now generated in Python)
% As the attack intersects lists of users that could potentially be messaging
% Alice, we expect that the 25\% of messages that are replies to messages from
% previous epochs have a minimal effect, as well as the 50\% of messages that are
% randomly generated. To test this we wrote a simplified simulation in Python,
% that uses a complete graph model (every user is connected to every other user)
% and communications in epochs after the first are only 25\% of repeated messages
% from previous epochs and 50\% of messages unrelated to previous epochs.

The breakdown of random/replies/repeats was chosen arbitrarily, and we
expect it to have a significant impact on the number of messages needed to
complete the attack (identify Alice). In particular, for Attack~1, the 25\% of
messages that are repeats from previous epochs will extend the time the
intersection of all epochs' recipients will be more than just Alice. We examined
the effect this percent has in \cref{fig:signal-percentageEffect}, finding that
larger percents of repeated messages require more messages to identify Alice as
the sender to Bob. We note that ``repeated'' messages in this context mean
messages that are correlated with Alice sending Bob a message. An example would
be that Dave frequently receives a message whenever Alice receives a message,
potentially because Dave is communicating with someone else in a similar pattern
as Alice and Bob. We expect such
unrelated cross-user message correlation to be relatively low in practice.

% \PercentageEffect

\subsubsection{Evaluating Attack~2}
\label{sssec:signal-attack}

\RankGraph

In Attack~2 (\cref{sec:signal-attack2}), there are potentially messages that Bob
receives that will not elicit a response to Alice, and we instead simply look
for the \emph{most common} user across all epochs, rather than the one that
appeared in all of them (as in Attack~1).

%As mentioned above, the previous subsections have dealt with a simplified model
%in which messages can be distinguished from receipts and Typing Notifications
%are disabled. We now examine the realistic model in which Alice and Bob enable
%Typing Notifications and all short messages, including receipts and Typing
%Notifications, are indistinguishable.

\LikelihoodGraph

For these simulations, we use the same graph generation methodology as before.
We generate messages similar to before,
with 25\% repeated message, 25\% response messages
and 50\% new random messages.  We also generate the same number of {\em
receipts}, simulated as replies from the previous epoch.

To simulate the possibility that Bob received a typing notification or another
kind of message that won't solicit a message to Alice, Bob
will not send a delivery receipt in every round. Similarly to model that other
users may send messages to Bob as well, some epochs are started by messages from
users other than Alice (and as such the delivery receipt is sent to them
instead). Instead, Bob sends a delivery receipt to Alice only with probability
$f$, which we vary from 10-90\%\footnote{$f = 100\%$ is equivalent to
Attack~1}.

%When Alice is having a conversation (repeated messages to each other) with Bob,
%Alice will show up routinely as the receiver of a sealed sender message shortly
%after Bob receives a message. Due to typing notifications being enabled there
%are some sealed sender messages that will be sent to Bob that do not prompt a
%delivery receipt from Bob to Alice. Bob may also be having a conversation with
%another Signal user. As such there will also be epochs which do not
%feature Alice. Thus the aforementioned intersection attack will inherently fail
%in this scenario. To address this an attacker can keep track of how many times
%each user receives a sealed sender message shortly after Bob does. This count
%will likely be very noisy at first, due to 579 messages being sent every epoch.
%However, Alice consistently messaging Bob will result in her eventually showing
%up as the receiver of a sealed sender message more often than other users not
%conversing with Bob.

Attack~2 counts the number of times each user identity
appears as a destination of a sealed sender message in each epoch after Bob
receives a message. A simple strategy is to wait for $N$ messages to Bob, and
then identify the user with the highest count as the person messaging Bob.
\cref{fig:signal-AliceLikelihood} shows the chance that an attacker will pick
Alice when using this strategy for 5 different values of $f$. We find that with
$f = 90\%$, it takes about 20 messages but with $f = 50\%$, it may take upwards
of 80 messages. For lower values of $f$ (as Bob receives more messages from
others or Alice sends multiple typing events per message), it takes even longer
to correctly identify Alice.


% TODO: make this about anonymity set
\cref{fig:signal-AliceRank} examines how Alice's (the correct sender) count
compares to other users. It effectively measures how the
``anonymity set'' of senders shrinks as more messages are sent to Bob. Even with
only a dozen messages, there are relatively few candidates that might have sent
Bob a message, which may be useful enough of a signal for an adversary to
identify the true sender given other information.

\subsubsection{Accounting for popular users}
In \cref{sec:signal-attack3}, there are users which are inherently popular
using the messaging service. Due to their popularity they provide a cover for
Alice, in that popular users will show up just as often as Alice does (or more
often if other users are messaging Bob, or Bob receives event messages).

To address this in simulations, we generate additional epochs worth of messages
where Bob has \emph{not} received a message. This provides a ``base rate'' of
activity on the service. These additional epochs are generated in between ``Bob
epochs'' to keep an updated pulse on the activity through the servers. The
results of the additional epochs are fed into (and subtracted from) the
\emph{counter} keeping track of who shows up most often. It is conceivable that
Alice might also be messaging another user between messaging Bob, and as such
show up in a non-``Bob epoch''. As such these additional epoch are taken at
random intervals so as not to provide a natural cover for Alice.

The number of epochs examined between ``Bob epochs'' can be modified by the
service provider as necessary to make appropriate accomodations for how often
popular users show up. Our simulations find success with generating only two
additional epochs for each ``Bob epoch''. The results can be examined in
\cref{fig:signal-variant3}.

\VariantThree

%As Alice may not be receiving a
%receipt the intersection attack will now fail to determine Alice as the sender,
%Figure~\ref{fig:AlicePercentile} examines how Alice's likelihood of being sender
%compares to other users as a function of time and
%Figure~\ref{fig:AliceLikelihood} examines the chances of success the attacker
%will have in determining that Alice is messaging Bob if they guess the user who
%shows up most often is the user conversing with Bob as a function of the number
%of messages Bob receives. 

%\PercentileGraph

% \RankGraph

\fi
