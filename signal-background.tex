\section{Background}\label{sec:Signal-background}

% Although many protocols have been suggested for sending end-to-end
% encrypted messages (see \cref{sec:relwork-msging}), the most commonly
% used in practice for mobile messaging is the Signal protocol~\cite{Mar13},
% currently used by both the Signal and WhatsApp applications. 

We now give
some background on the structure and types of messages in the Signal protocol~\cite{Mar13}, used in both the Signal and WhatsApp applications.

\subsection{Sealed Sender Messages} % what is signal, what is sealed sender
\label{ssec:signal-signal}
Although % the protocols used by
secure end-to-end encrypted messaging %(SEEM)
applications like Signal protect the
contents of messages% sent between users
, they reveal metadata about
\emph{which} users are communicating to each other. In an attempt to hide this
metadata, Signal recently released a feature called sealed
sender~\cite{sealed-sender} that removes the sender from the metadata
intermediaries can observe.

To send a sealed sender message to Bob, Alice connects to the Signal server and
sends an encrypted message to Bob anonymously\footnote{As we note in our threat
model, we do not consider the information leakage from networking.}. Within the
payload of this encrypted message, Alice includes her own identity, which allows
Bob to authenticate the message.  Importantly, Signal still learns Bob's
identity, which is needed in order to actually deliver it. %; without this identity, Signal cannot deliver the message.
The structure of sealed sender messages are illustrated in \cref{fig:signal-SealedSenderStructure}.

Due to sender anonymity,
Signal cannot directly rate-limit users to prevent spam or abuse.  Instead,
Signal derives a 96-bit \emph{delivery token} from a user's profile key, and
requires senders demonstrate knowledge of a recipients' delivery token to send
them sealed sender messages.
% For instance, to send a sealed sender message to
% Bob, Alice must have knowledge of his delivery token.
By only sharing this delivery token with his contacts, Bob
limits the users who can send him sealed
sender messages, thus reducing the risk of abuse\footnote{There are a number of
options available to Bob that can allow more fine-grained access control to his
delivery token.  Bob can opt to receive sealed sender messages from anyone even
without knowledge of his delivery token, but this is disabled by default.
Additionally, Bob can regenerate his delivery token and share it only with a
subset of his contacts to block specific users.}.

\subsection{Types of Messages}
\label{sec:signal-message-types}

We manually reviewed and instrumented the Signal messenger Android
4.49.13 source code~\cite{signal-source} in order to understand the types of
messages Signal sends. In addition to the messages that contain content to be
delivered to the receiver, there are several event messages that can be sent
automatically, as discussed below. All of these messages are first padded to the
next multiple of 160 bytes, then encrypted and sent using sealed sender (if
enabled), making it difficult for the Signal service to distinguish events from
normal messages based on their length. %Figure~\ref{fig:SealedSenderStructure}
%shows the structure of sealed sender messages, and Figure~\ref{fig:message-stages}
%shows the user interface for each stage of message acknowledment in Signal.

\SealedSenderStructure

\MessageStages

\medskip \noindent
\textbf{Normal message.}
A normal text message or multimedia image sent from Alice to Bob is the
typical message we consider. A short (text) message will be padded to
160 bytes, and longer messages padded to a multiple of 160 bytes, before
encryption.
% could cut?
%While the application UI displays a stream of messages
%between Alice and Bob in a single window as a conversation, in the
%protocol there is no distinction between the first or any subsequent
%messages in a given conversation.


\medskip \noindent
\textbf{Delivery receipt.} 
When Bob's device \emph{receives} a normal message, his device will
automatically send back a delivery receipt to the sender. When Alice receives
the delivery receipt for her sent message, her device will display a second
check mark on her sent message to indicate that Bob's device has received the
message (see \cref{fig:signal-message-stages}). If Bob's device is online when Alice sends her message, the delivery receipt will be sent back immediately. 
We measured a median time of 1480~milliseconds between sending a message and receiving a delivery receipt from an online device. (See Figure~\ref{fig:signal-delivery-cdf} for CDF of times.)
These receipts
{\em cannot be disabled} in Signal.
\DeliveryTimeCDF
% 90% of messages received the delivery receipt in less than 1880ms

\medskip \noindent
\textbf{Read receipt (optional).} 
Bob's device will (optionally) send a read receipt to the sender when he has \emph{viewed} a normal
message, triggering a UI update on Alice's device 
(see \cref{fig:signal-message-stages}). Unlike delivery receipts, Bob can disable read receipts. However, Alice may still send read receipts for messages she receives
from Bob. If Bob receives a read receipt but has the feature disabled, his user
interface will not display the notification. %When interacting with the UI, the
%feature will appear to be disabled for both users if Bob turns it off, although
%Alice might still be sending read receipts.

\medskip \noindent
\textbf{Typing notifications (optional).}\label{sssec:signal-typingNotifications}
Alice's device will (optionally) send a \emph{start typing} event when Alice is entering a
message, which Bob's device will use to show that Alice is typing. If she does
not edit the message for 3 seconds, a \emph{stop typing} event will be sent.
Each sent message is accompanied by a \emph{stop typing} event to clear the
receiver's typing notification. Like read receipts, typing notifications can be
disabled such that the user will not send or display received notifications.
%However, if their peer still has typing notifications enabled, their device will
%still receive (but ignore) these events.


% More intro-y than background-y
%Signal~\cite{signal} is a popular secure end-to-end encryption smartphone app
%that uses off-the-record~\cite{otr} communication to protect the contents sent
%between users. 

\subsection{Threat Model}
\label{sec:signal-threat}
% assumptions of attack, we know that a Network
% inference attack is possible no matter what, assume that messages/receipts are
% sent through Tor.

We assume that the service provider (e.g. Signal) passively monitors messages
to determine which pairs of users are communicating.  This models either an insider threat or a service provider compelled to perform surveillance in response to a government request.
We assume Alice and Bob have already exchanged delivery tokens and they communicate using sealed sender. Once initiated, we
assume that Alice and Bob will continue to communicate over time.  Finally, we also assume that
many other users will be communicating concurrently during Alice and Bob's conversation,
potentially with Alice and/or Bob.  
%\gsk{is this the
%goal?  like is the output of adv 1/0 or is it an identity bob}.

The service provider cannot view the contents of the encrypted sealed sender
messages, but knows the destination user for these messages (e.g. someone sends
a message to Bob).  We assume that Alice and Bob have verified their respective 
keys out of band,
and that the applications/devices they are using are secure.
Although the service provider publishes the application, they typically
distribute open-source code with deterministic builds, which we assume prevents
targeting individual users.

We note that the service provider could infer a sender's identity from network metadata
such as the IP address used to send a sealed sender message. However, this is a problem
that could be solved by using a popular VPN or an
anonymizing proxy such as Tor~\cite{signal-tor, tor_two}. For the purposes of
this paper, we assume that users who wish to remain anonymous to Signal can use
such proxies (e.g. Orbot~\cite{guardianproject}) when sending sealed sender messages (and, in our solution,
when receiving messages to ephemeral mailboxes), and we do not use network
metadata in our attack.

In terms of impact, we note that a recent study
suggests as many as 15\% of mobile users already use VPNs every day
\cite{vpn-usage};
this prevalence is even higher in east Asia and, presumably, among
vulnerable user populations.
%When receiving
%messages, users are unlikely to benefit from Tor, as they must authenticate to
%Signal with their
%identity to receive messages.

%%% Local Variables:
%%% mode: latex
%%% TeX-master: "main"
%%% End:
