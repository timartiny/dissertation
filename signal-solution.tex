\section{Solutions}\label{sec:signal-solution}
We now present three protocols that follow the security definition from
\cref{sec:signal-secdef} and, in particular, prevent the attacks presented in \cref{sec:signal-attack}.  We first outline a {\em one-way} sealed sender conversation in \cref{sec:signal-levelonesolution}, in which the initiator of the conversation remains anonymous.  We prove that our construction meets the definition presented in \cref{sec:signal-apxdefinitions}.   In \cref{sec:signal-leveltwosolution}, we extend this protocol to give better privacy to the receiver using a {\em two-way} sealed sender conversation.  Finally, in \cref{sec:signal-levelthreesolution}, we address denial of service attacks that malicious users could launch against the server. 

\medskip \noindent
\textbf{Overview of Solutions.} Our key observation is that the attack described in \cref{sec:signal-attack} is only possible because both users in a conversation are sending messages to the other's long-term identity.  Over time, these messages can be correlated, revealing the identities of the users.
On the other hand, if {\em anonymous} and {\em ephemeral} identities are used instead, then user's true identities can remain hidden.  However, anonymous identities lead to a bootstrapping problem: {\em how do users initiate and authenticate a conversation if they are using fresh, pseudonyms?} 

In a {\em one-way sealed sender conversations}, the identity of one side
of the conversation is leaked, namely the initial message receiver, in
order to solve this bootstrapping problem. This closely models the
situation of a whistle-blower, where the informant wishes to stay
anonymous, but the reporter receiving the information can be public. At
a high level, the initiator of the conversation begins by creating a
fresh, anonymous identity and then sends this identity to a receiver via
a normal sealed sender message (thus solving the bootstrapping problem).
The conversation proceeds with the initiator of the conversation sending
messages to the receiver using sealed sender (one way), and the
conversation receiver sending replies to the initiator's anonymous
identity. Importantly, the identity of the initiator is never leaked, as
no messages exchanged in the conversation contain that person's long-term
identity.  We prove that out protocol securely realizes the definition
of sealed sender conversations presented in
\cref{sec:signal-apxdefinitions}. 

A straightforward extension is to move towards {\em two-way sealed
sender conversations} where both parties use anonymous identities. This solution is described
in \cref{sec:signal-leveltwosolution}. When an initiator
starts a conversation as described above, the receiver
also creates a new anonymous identity and sends it via sealed sender back to the
conversation initiator. This protocol offers a single opportunity to
link the receiver to their new, anonymous identity (by correlating the
timing of the received message and the registering of a new public key),
but, as we have shown, network noise makes it difficult to re-identify
users with only a single event. Even in the unlikely case that
the conversation receiver is
linked to their long-term identity, we show that the conversation initiator
remains anonymous.

Both protocols place the service provider at risk of denial of service
attacks, and so in \cref{sec:signal-levelthreesolution}, we aim to
limit the power of users to arbitrarily register anonymous identities.
Allowing users to create unlimited anonymous identities would lead to
strain on the service provider if there is no way to differentiate
between legitimate anonymous identities and malicious ones. To prevent
these attacks, users are each given a limited number of anonymous
credentials that they can ``spend'' to register anonymous keys,
reminiscent of the earliest e-cash systems \cite{C:Chaum82}.  These
credentials can be validated by the service provider to ensure that a
legitimate user is requesting an anonymous identity without revealing
that user's identity. We use blind signatures to implement our anonymous
credentials.  We evaluate the practicality of this approach in
\cref{sec:signal-impl} and show that it could be deployed cheaply for
either one-way or two-way sealed sender conversations.


For simplicity, we assume that communicating users have already exchanged delivery tokens.  Any protections derived from these delivery tokens can be added to the following protocols in a straightforward manner.  Additionally, we assume users connect to the service provider via an anonymous channel, e.g., Tor or Orbot.

% that
% Alice uses a new anonymous identity in her conversation
% with Bob instead of her own, public identity.
% One can think of this ephemeral identity as a mailbox that is opened
% anonymously in a delivery center. Bob's messages back to Alice can
% then be addressed to this mailbox without revealing Alice's identity.

% Note that only one party in a two-way conversation (Alice, in our story)
% uses an anonymous mailbox; the other party (Bob) uses his public
% mailbox. This is a natural extension of the existing sealed sender
% protocol upon which our solution is based, which reveals the recipient
% of each message but not its sender. Our solution extends this idea of
% single-sided privacy from the level of individual messages (which the
% sealed sender protocol provides) to the level of multiple-message
% conversations (which our attack demonstrates sealed sender alone does
% not provide).

% When beginning a new conversation with Bob, Alice first creates a new
% ephemeral identity, and this ``return address'' is included with her
% sealed-sender message to Bob. When Bob replies (either with a delivery
% receipt, read receipt, or an actual response), he sends a sealed-sender
% message to the anonymous mailbox address that Alice sent him.
% In this way, even a powerful attacker who can use timing to connect
% messages and responses will only see that Bob is in a conversation with
% some anonymous mailbox, without any possibility of learning Alice's
% identity.

% Like in the sealed sender protocol itself, the main technical challenge
% in our solution is preventing the abuse that anonymity would otherwise
% enable; that is, preventing users from registering too many anonymous
% mailboxes.
% We limit the number of mailboxes per user, without compromising
% anonymity, by using anonymous credentials
% implemented with blind signatures.  Each user is issued a fixed number of anonymous
% credentials, each of which can be swapped for an active mailbox.  Once
% that mailbox is no longer needed, it can be closed and the user can be issued a fresh
% anonymous credential, all without the possibility of linking any user to
% any anonymous mailbox.


\subsection{Preliminaries}

\noindent
\textbf{Sealed Sender}
We assume that the service provider implements the sealed sender
mechanism described in \cref{ssec:signal-signal}.  Specifically, we assume that a client can generate a public/private key
pair and publish their public key as an address registered with the
service. If the server permits it through some verification process, the
server will allow messages to be sent to that public key without a sender.

More formally, we assume that the system has a sealed sender encryption scheme $\sealedsenderencryptionscheme$.  While Signal does not give a proof of security for the scheme it uses, for our constructions we will assume that $\sealedsenderencryptionscheme$ is a signcryption scheme that satisfies ciphertext anonymity \cite{PKC:LibQui04} and adopt the notation presented in \cite{ACISP:WMAS13} for its algorithms\footnote{We note that ciphertext anonymity is actually a stronger primitive than required, as there is no need for receiver anonymity.}. We say a sealed sender encryption scheme $\sealedsenderencryptionscheme$ is a set of three algorithms:
% $(\sealedsenderencryptiongen, \sealedsenderencryptionenc, \sealedsenderencryptiondec, \sealedsenderencryptionverify)$ defined as follows:
\begin{itemize}[noitemsep]
  \item $\sealedsenderencryptiongen(1^\secparam) \rightarrow (\pk, \sk)$ generates a public/private key pair.

  \item $\sealedsenderencryptionenc(m, \sk_s, \pk_r) \rightarrow c$
  takes in a message $m$, the sender's secret key $\sk_s$ and the
  receiver's public key $\pk_r$, and outputs a ciphertext $c$

  \item $\sealedsenderencryptionvdec(\sk_r, c) \rightarrow \{
  (m,\pk_s), \bot \}$ takes in the receiver's private key $\sk_r$ and
  a ciphertext $c$ and either outputs a message $m$,and the public key of the sender $\pk_s$,
  or returns the error symbol $\bot$.
  (Note that this actually constitutes decryption followed by verification
  in the notation of \cite{ACISP:WMAS13}, returning $\bot$ when either step fails.)

%   \item $\sealedsenderencryptiondec(\sk_r, c) \rightarrow \{
%   (m,x,\pk_s), \bot \}$ takes in the receiver's private key $\sk_r$ and
%   a ciphertext $c$ and either outputs a message $m$, some additional
%   information $x$, and the public key of the sender $\pk_s$, or returns the error symbol $\bot$. 
% 
%   \item $\sealedsenderencryptionverify(m,x,\pk_s) \rightarrow \bin$
%   takes in the output of a decryption and verifies if it is correct.
\end{itemize}
Formal security definitions are given in \cite{ACISP:WMAS13}.  In short, the scheme satisfies (1) message indistinguishability, (2) unforgeability, and (3) ciphertext anonymity, meaning the ciphertext reveals nothing about the sender or receiver. 


% \paragraph*{Delivery Token.} Additionally, we assume that there is some sealed sender authentication mechanism $\sealedsenderauthscheme = (\sealedsenderauthschemegenerate, \sealedsenderauthschemeauth, \sealedsenderauthschemeverify)$ that allows a sender to prove that they are allowed to send a sealed sender message to a specific receiver.  We require that the output of $\sealedsenderauthschemeauth$ does not reveal the identity of the sender.  This scheme replicates the delivery tokens in Signal (see Section \ref{}).

\medskip \noindent
\textbf{Blind Signatures}
The mechanism to prevent abuse for the creation of anonymous accounts relies on the 
cryptographic primitive of \emph{blind signatures}, as first proposed by
\cite{C:Chaum82}. Blind signature schemes
have 5 algorithms: \bskeygen, \bsblind, \bssign, \bsextract and \bsverify.  
\bsblind takes in the public key of the signer, a message, and some randomness and
outputs a blinded message.% and a blinding factor.
\ \bssign takes in
the signer's private key and a blinded message and outputs a 
blinded signature.  \bsextract takes in
%the blinding factor produced by \bsblind and
a blinded signature and the randomness used in blinding and outputs a normal 
signature.  Finally, \bsverify takes in a message and the 
signer's public key and decides if the signature is valid.

% following steps between a client (message generator) and server
% (signer).

The interaction between a server with the signing keypair $sk, pk$ and a client is as follows:
\begin{enumerate}[noitemsep]
  \item Client generates the blinded message % and blinding factor $(b, bf)
  \\ $b \gets \bsblind(m, pk;r)$ for $r \sample \bin^\securityparameter$
  \item Client sends $b$ to the server for signing.
  \item Server computes the blinded signature 
  \\ $s_{blind} \gets \bssign(b,sk)$ and returns it to the client.
  \item Client extracts the real signature % $s = \bsextract(s_{blind}, bf, pk)$
  \\ $s \gets \bsextract(s_{blind}, pk; r)$
  \item Client, in a different network connection, sends the initial message $m$ and the real signature $s$ to the server, who runs $\bsverify(pk,m,s)$ 
\end{enumerate}
The blind signature scheme should have the usual signature
unforgeability property.  Additionally, it should be impossible for a
server to link the blinded message and blinded signature to the real
message and real signature.  We use the RSA-based construction of blind
signatures from \cite{C:Chaum82}.
%%% Local Variables:
%%% mode: latex
%%% TeX-master: "main"
%%% End:


\subsection{One-way Sealed Sender Conversations}\label{sec:signal-levelonesolution}

First, we provide the construction of sealed sender conversations which we build on in this solution and those that follow.
%
Recall that a sealed sender conversation reveals the flow of the conversation (including message timing, etc.) and the identity of the initial receiver, but at no point can the service provider identify the initial sender.

The intuition behind our solution is straightforward: when initiating a new conversation, a sender generates an ephemeral, per-conversation key pair.  This key pair is registered with the service provider anonymously%
% (using some form of anonymizing proxy such as Tor)
, but otherwise is treated as a normal identity in the system.  Throughout the lifetime of the conversation, this identity key is used instead of the long-term identity of conversation initiator.  As long as the ephemeral public key is never associated with the long-term identity, and never used in any other conversations, the service provider cannot learn anything about the true identity of the user that generated that ephemeral identity.

Generally, the flow of a sealed sender conversation is as follows.  During the setup, each sender $\senders$ with long-term keys $(\pk_s, \sk_s)$  creates entries $(\receiverr, \pk_r, \pk_s)$ for each receiver $\receiverr$ with public key $\pk_r$.  Some user, who we call the initiator, starts the conversation by running the {\bf Initiate Conversation} protocol below where $\senders$ generates and registers an ephemeral identity for a receiver $\receiverr$.  Whenever the receiver comes online (or possibly immediately by receiving a push notification) and receives the appropriate information, they will locally associate the ephemeral key with the initiator for the duration of the conversation.  From this point, both users may send messages using the {\bf Send Message} protocol and receive those messages from the service provider via {\bf Push Message}, over an open, long-term connection. 
The protocol {\bf Open Receiver Connection} is used to establish a
channel for such push notifications, either for a user's long-term
mailbox, or for an ephemeral mailbox created for a single conversation.

Every user must maintain a \emph{conversation table}, to remember where
messages should be sent in all ongoing conversations. Each table entry
stored by a user $\senders$ is a tuple
$(\receiverr, \pk_\beta, \pk_\alpha, \sk_\alpha)$, where $\receiverr$ is the actual
message recipient, $\pk_\beta$ is the recipient's mailbox (public key) to
which the message is addressed, and $(\pk_\alpha, \sk_\alpha)$ is the key pair used
to sign and encrypt the message. Depending on who initiated the
conversation, one of $\pk_\beta$ or $\pk_\alpha$ will correspond to an
ephemeral identity $\pk_e$, and the other will correspond to one of the long-term identities
$\pk_r$ or $\pk_s$.

% Each side of the conversation $\senders$ and $\receiverr$ must maintain a conversation table. $\senders$ stores $(\receiverr, \pk_r,\pk_e,\sk_e)$ where $(\pk_e,\sk_e)$ are ephemeral public and secret keys used for the conversation with $\receiverr$. The receiver $\receiverr$ creates an entry $(\senders, \pk_e, \pk_r,\sk_r)$ that associates the sender with their long-term public key and the ephemeral public key. 

% $$(\texttt{User}, \texttt{User's } \pk, \texttt{My } \pk, \texttt{My } \sk)$$

\medskip \noindent
\textbf{Initiate One-Way Sealed Conversation to $\receiverr$:}
\begin{enumerate}
  \item Initiator $\senders$ does the following:
  \begin{enumerate}[nosep]
  \item looks up $\receiverr$'s long-term public key $\pk_r$
    %and authentication token $\deliverytoken_r$
    \item generates fresh ephemeral keys $(\pk_e, \sk_e) \leftarrow \sealedsenderencryptionscheme.\sealedsenderencryptiongen(1^\securityparameter)$% and authentication token $\deliverytoken_e \leftarrow \sealedsenderauthscheme.\sealedsenderauthschemegenerate(1^\securityparameter)$.
    \item encrypts $c \leftarrow \sealedsenderencryptionscheme.\sealedsenderencryptionenc(\texttt{``init''} \| \pk_e, \sk_s, \pk_r)$
    % \item encrypts $c \leftarrow \sealedsenderencryptionscheme.\sealedsenderencryptionenc(\texttt{``init''} \| \pk_e \| \deliverytoken_e, \sk_s, \pk_r)$
    % \item generates the authentication message $a \leftarrow \sealedsenderauthscheme.\sealedsenderauthschemeauth(\pk_r, \deliverytoken_r)$
    \item connects to the service provider anonymously and sends $c \| \pk_e$ for $\pk_r$ 
    \item appends $(\receiverr, \pk_r, \pk_e, \sk_e)$ to the conversation table
    % \item record $(\pk_e, \sk_e), \deliverytoken_e$ as a sender identity   
    \item Registers a new mailbox for the public key $\pk_e$ and uses \textbf{Open Receiver Connection} with keypair public key $\pk_e, \sk_e$ to establish a connection for push notifications.
  \end{enumerate}

  \item The service provider delivers $c$ (sealed sender) to $P_r$ based on $\pk_r$, either immediately pushing the message or waiting for the receiver to come online.

  \item When the receiver $\receiverr$ receives the message to its long-term mailbox
    $\pk_r$, it:
    %When the receiver calls receive message,
  \begin{enumerate}[nosep]
    \item decrypts and verifies\\ $(\texttt{``init''} \| \pk_e, x, \pk_s)  \leftarrow
    \sealedsenderencryptionscheme.\sealedsenderencryptionvdec(\sk_r, c)$
%     \item verifies
%   $\sealedsenderencryptionscheme.\sealedsenderencryptionverify(\texttt{``init''} \|
%   \pk_e, x,\pk_s) =1$
    \item appends $(\senders, \pk_e,
  \pk_r, \sk_r)$ to the conversation table
    \item uses {\bf Send Message} to send a delivery receipt to $\senders$ (which now goes to $\pk_e$ from the conv.\ table)
  \end{enumerate}
  % \item When the receiver calls receive message, it decrypts $(\texttt{``init''} \| \pk_e \| \deliverytoken_e, x, \pk_s)  \leftarrow \sealedsenderencryptionscheme.\sealedsenderencryptiondec(\sk_r, c)$ and verifies $\sealedsenderencryptionscheme.\sealedsenderencryptionverify(m \| \pk_e \| \deliverytoken_e, x,\pk_s) =1$.  The receiver then records $(\sender, \pk_e, \deliverytoken_e, \pk_r, \deliverytoken_r)$ and uses {\bf send message} to send an acknowledgment to $\pk_e$ 
\end{enumerate}

\medskip \noindent
\textbf{Send Message to $\personx$}
\begin{enumerate}[noitemsep]
  \item Sender looks up freshest entry $(\personx, \pk_\beta,
  \pk_\alpha, \sk_\alpha)$
  in the conversation table.%
  % \footnote{Note that one of $\pk_\alpha$ and $(\pk_\beta,\sk_\beta)$ will be ephemeral and the other will be long-term, depending on which party initiated the conversation.}

  \item Sender encrypts $c \leftarrow \sealedsenderencryptionscheme.\sealedsenderencryptionenc(m, \sk_\alpha, \pk_\beta)$

  % \item Sender generates the authentication message required to prove that they are allowed to send to the receiver $a \leftarrow \sealedsenderauthscheme.\sealedsenderauthschemeauth(\pk_r, \deliverytoken_r)$

  % \item Sender sends $a \| c$ to the service provider, anonymously if necessary, to send to $\receiverr$
  \item Sender sends $c$ for $\pk_\beta$ to the service provider,
  anonymously if necessary.%, for receiver $\receiverr$.  
  \item If there is an open connection associated with $\pk_\beta$, the service provider uses \textbf{Push Message} for $c$ over that connection. Otherwise, the service provider sets the message as pending in the mailbox associated with $\pk_\beta$
  % \item The service provider verifies $\sealedsenderauthscheme.\sealedsenderauthschemeverify(a, \pk_r)=1$, and drops it otherwise.  The service provider sets the message as pending in the mailbox associated with $\pk_r$

\end{enumerate}

\medskip \noindent
\textbf{Open Receiver Connection for $(\pk_\beta,\sk_\beta)$}
\begin{enumerate}[noitemsep]
  %\item Receiver $P_s$ looks up entry $(P_r, \pk_s, \pk_e, \sk_e)$ in the conversation table.
  \item Receiver connects to the service provider and demonstrates
  knowledge of key pair $(\pk_\beta,\sk_\beta)$ such that there is a registered mailbox for public key $\pk_\beta$
  \item The receiver and the server build a long-term connection for message delivery, indexed by $\pk_\beta$
  \item If there are any pending messages in the mailbox associated with $\pk_\beta,$ use \textbf{Push Message} for those messages.
    %\item $\pk'_s=\pk_s$
    %\item $\sealedsenderencryptionscheme.\sealedsenderencryptionverify(m_i,x,\pk_\alpha)=1$
\end{enumerate}

\medskip \noindent
\textbf{Push Message $c$ to $\pk_\beta$}
\begin{enumerate}[noitemsep]
  %\item Receiver $P_s$ looks up entry $(P_r, \pk_s, \pk_e, \sk_e)$ in the conversation table.
  \item Service provider looks up an open connection indexed by $\pk_\beta$.  If such a connection exists, the service provider sends $c$ over it
  \item Receiver decrypts $c$ as $(m,\pk_\alpha) \leftarrow \sealedsenderencryptionscheme.\sealedsenderencryptionvdec(\sk_\beta, c)$ and verifies
    an entry $(\personx, \pk_\alpha,\allowbreak \pk_\beta, \sk_\beta)$ exists in the conversations table,
    dropping it otherwise.
    %\item $\pk'_s=\pk_s$
    %\item $\sealedsenderencryptionscheme.\sealedsenderencryptionverify(m_i,x,\pk_\alpha)=1$
\end{enumerate}


We prove that this construction securely realizes the definition \cref{fig:signal-definition} in the standalone model in \cref{sec:signal-proof}.  The proof is straightforward: we construct a simulator and show that an adversary corrupting the service provider and any number of clients cannot distinguish between the real protocol and interacting with the ideal functionality.



%%% Local Variables:
%%% mode: latex
%%% TeX-master: "main"
%%% End:


\subsection{Two-way Sealed Sender Conversations}
\label{sec:signal-leveltwosolution}


While the construction above successfully realizes sealed sender
conversations,
the identity of the receiver is still leaked to
the service provider.  Ideally, we would like for both users in a
conversation to communicate using only ephemeral identities, so that
the service provider sees only the flow of messages in a conversation
but does not learn either party's long-term identity.
However, this again leads to a bootstrapping problem: if
both users use fresh, anonymous identities, {\em how do they exchange
this ephemeral contact information while remaining anonymous?} 

While heavyweight cryptography (such as PIR or ORAMs)
may provide a more robust solution,
in this work we focus on scalable solutions that might plausibly be
adopted by secure messaging platforms.  As such, we present a natural
extension of our one-way sealed sender conversation protocol.

After an initiator creates an ephemeral key pair, opens a new mailbox,
and sends this to the
receiver, the receiver responds by doing the same thing: creating a
second ephemeral key pair, opening a second mailbox, and sending this
back to the initiator as part of the initial delivery receipt. After
this, \emph{both} the conversation initiator and receiver will have
conversation table entries of the form
\(\personx, \pk_{e1}, \pk_{e2}, \sk_{e2}\), with two different ephemeral
keys for sending and receiving messages in the conversation.

This requires minimal changes to the previous protocol. Essentially, the
\textbf{Initiate} protocol gains another section for the recipient to
create their own ephemeral identity, but the \textbf{Send},
\textbf{Open Connection}, and \textbf{Push Message} protocols are identical. In \cref{sec:signal-leveltwodetails}
we provide the full details of these updated protocols, along with an
additional protocol \textbf{Change Mailbox} which is used to update an
ephemeral key pair for one side of an existing conversation.

\medskip \noindent
\textbf{Security.}  We have two security goals for this protocol.  First, we require that this protocol is a secure instantiation of a one-way sealed sender conversation, just like the protocol above.  This is clear, as the only party whose behavior changes from the protocols in
\cref{sec:signal-levelonesolution} is the initial receiver.  Simulating their behavior is easy because that user's identity is already leaked by the ideal
functionality. As such, the proof remains nearly identical to that in
\cref{sec:signal-proof}.

Second, we require that the service provider has only one chance to identify the initial receiver.  Note that besides the initial messages, all sent messages are only linked to the anonymous identities.  Thus, no information about the users' true identities are leaked by these messages.  This only source of information about these identities comes from the timing of the mailbox's initial opening, so this is the only chance to identify the initial receiver.   
As described in our simulations, in a reasonably busy network it is difficult to link two events perfectly.  Instead, it requires many epochs of repeated behavior to extract a link.  Therefore, giving the service provider only a single chance to de-anonymize the receiver will most likely (\emph{though not provably}) provide two-sided anonymity.
To further decrease the chance of a successful attack, the initial receiver can introduce some initial random delay in opening and using a new mailbox.%, making it harder for the service provider to extract this link.

\medskip \noindent
\textbf{Obscuring the Conversation Flow.} A natural generalization of this approach is to switch mailboxes often throughout a conversation, possibly with {\em each message}.
%That is, when a user runs the send message subprotocol, they also run the change mailbox protocol at the same time.  
%This would be an attempt at obscuring the conversation flow from the service provider,
This may provide further obfuscation, as each mailbox is only used once.  While analyzing {\em how well} this approach would obscure the conversation flow is difficult, as linking multiple messages together requires the service provider to find a timing correlation between the various mailboxes' activities, it is clear it provides no worse anonymity than the above construction.
%Finally, if opening a mailbox requires anonymous credentials (see next section), this approach might require allowing users to close existing mailboxes in exchange for a new anonymous credential that can be used later.



\subsection{Protecting against Denial of Service}\label{sec:signal-levelthreesolution}

Both constructions presented above require users to anonymously register
public keys with the service provider.  This provides an easy way for
attackers to launch a denial of service attack% on the service provider
: simply anonymously register massive numbers of public keys.
%Importantly, there is no way for the service provider to remove malicious users, as their identity cannot be determined.  It is unlikely that a service provider would be willing to deploy either one-way or two-way sealed sender conversations at the cost of enabling this attack.
As such, we now turn our attention to bounding the number of ephemeral identities a user can have open, without compromising the required privacy properties.

We build on {\em anonymous credential} systems, such as \cite{C:Chaum82}.  Intuitively, we want each user in the system to be issued a fixed number of anonymous credentials, each of which can be exchanged for the ability to register a new public key. To implement this system, we add two additional protocols to those presented above%
%, which we describe below
: {\bf Get signed mailbox key} and {\bf Open a mailbox}.

In {\bf Get signed mailbox key}, a user $P_s$ authenticates to the service provider with their long-term identity $\pk_s$ and uses a blind signature scheme to obliviously get a signature $\sigma_{es}$ over fresh public key $\pk_{es}$.  We denote the service provider's keypair $(\pksign, \sksign)$.  In {\bf Open a mailbox}, a user $P_s$ anonymously connects to the service provider and presents $(\pk_{es}, \sigma_{es})$.  If $\sigma_{es}$ is valid and the service provider has never seen the public key $\pk_{es}$ before, the service provider opens a mailbox for the public key $\pk_{es}.$  These protocols are described below:

\medskip \noindent
\textbf{Get signed mailbox key}
\begin{enumerate}[noitemsep]
  \item User authenticates using their longterm public key.  Server checks that the client has not exceeded their quota of generated ephemeral identities.
  \item Client generates $(\pk_e, \sk_e) \leftarrow \sealedsenderencryptionscheme.\sealedsenderencryptiongen(1^\securityparameter)$ 
  \item Client blinds the ephemeral public key $b \leftarrow \bsscheme.\bsblind(\pk_e, \pksign; r)$ with $r \leftarrow \bin^{\secparam}.$
  \item Server signs the client's blinded public key with $s_{blind} \leftarrow \bsscheme.\bssign(b,\sksign)$ and returns the blinded signature to the client.
  \item \label[step]{getsigned:extract}
    Client extracts the real signature locally with $\sigma_e \leftarrow \bsscheme.\bsextract(s_{blind}, \pksign; r)$
\end{enumerate}

\medskip \noindent
\textbf{Open a mailbox}
\begin{enumerate}[noitemsep]
  \item Client connects anonymously to the server and sends $\pk_e, \sigma_e$
  \item Server verifies $\bsscheme.\bsverify(\pksign, \sigma_e) = 1$ and checks $\pk_e$ has not been used yet.  
  \item Server registers an anonymous mailbox with key $\pk_e$ with an expiration date.
\end{enumerate}

Integrating these protocols into one-way sealed sender conversations and two-way sealed sender conversations is straightforward.  At the beginning of each time period ({\em e.g.} a day), users run {\bf Get signed mailbox key} up to $k$ times, where $k$ is an arbitrary constant fixed by the system.  Then, whenever a user needs to open a mailbox, they run the {\bf Open a mailbox} protocol.  Sending and receiving messages proceeds as before.

It is important that (1) the signing key for the blind signature scheme public key $\pksign$ be updated regularly, and (2) anonymous mailboxes will eventually expire.  Without these protections, malicious users {\em eventually} accumulate enough anonymous credentials or open mailboxes that they can effectively launch the denial of service attack described above.  Additionally, each time period's $\pksign$ must be known to all users; otherwise the server could use a unique key to sign each user's credentials, re-identifying the users.


\subsection{Blind Signature Performance} % timing, about this implementation (ripping parts out from more complicated protocols)
\label{sec:signal-impl}

To test the feasibility of using blind signatures, we implemented the
protocols in \cref{sec:signal-levelthreesolution} for a single client and
server. This represents the \emph{cryptographic overhead} of applying
our solution, as the remainder (sending and receiving messages,
registering keys) are services already provided by Signal.
%We now test the feasibility of using blind signatures to limit the number of anonymous mailboxes that can be opened by users.  We do this by implementing a server that runs the protocols specified in Section~\ref{sec:levelthreesolution}.  These protocols are the overhead required to run the solutions we have described, as sending and receiving messages and authenticating clients are already implemented.

\begin{table}
\centering
  \begin{tabular}{l|ccc}
      \toprule
     ~~Network  & ECDSA  & Get Signed  & Open a \\
    ~Conditions & KeyGen & Mailbox Key & Mailbox \\
    \midrule
    End-to-End & 0.049 &    0.061   & 0.039 \\ 
    User Local & 0.049 &    0.032   & 0.024 \\
    Server Local & N/A &    0.013   & 0.001 \\
    \bottomrule
  \end{tabular}
        \caption{Timing results (in seconds) for protocols of
        \cref{sec:signal-levelthreesolution}, using RSA-2048 ciphertexts and ECDSA.
        %WAN times were taken with a server, running on AWS, and client on the same coast of the US.  Local results are the same operations with the server and client running locally on the same computer.
        }
  \label{tbl:signal-timing_results}
\end{table}


The networking for both the client and server are written in
Python, with the Django web framework \cite{django} on the server.
Starting with the code provided in \cite{blind_rsa_old_code}, we
implement an RSA-2048 blind signature \cite{C:Chaum82} library in Java that can be called via RPC. Although RSA ciphertexts are large, they are very fast
to compute on modern hardware.%  The Python code, both on the client and
%server side, uses this RPC interface to blind, sign, unblind, and verify
%signatures.  We use this RPC interface to ensure that the JVM can be
%initialized before execution and long-term keys can be loaded.

We evaluated our implementation by running the server on an AWS instance
with 2 Intel Xeon processors and 4 GB of RAM.  The client was running on
a consumer-grade laptop, with a 2.5 GHz Intel i7 with 16 GB of RAM,
located in the same region as the AWS server.
%Our client and server were both located on the same coast of the United States.
We report the timing results in \cref{tbl:signal-timing_results} for each
protocol.
%, divided into three categories.  First, we report the time it takes to generate the ephemeral key material for the sealed sender encryption scheme ({\bf Get Signed Mailbox Key} Step 2), in this case an
% ECDSA key.  Second, we report how long it takes for the client to blind the public key and interactively get a valid signature ({\bf Get Signed Mailbox Key} Step 3-5).  Finally, we report the time it takes to open a new mailbox, including verifying the signature ({\bf Open a mailbox}).
To better isolate the overhead from network delay, we also report the
execution time when server and client are running locally on the same
machine.

Importantly, ECDSA KeyGen can be run in the background of the client,
long before the interactive phase of the protocol starts.  For maximum
security, a user may close an old mailbox and get a new signed key (with the same anonymous connection), and then
open a new mailbox with each message that they send.  This incurs an
overhead of less that 100ms, even including network delay.  The communication overhead of running this 
full protocol is less that 1KB, constituting 3 RSA-2048 ciphertexts and 1 ECDSA public key.

\subsection{Deployment Considerations}
\label{sec:signal-deployment}

\medskip \noindent
\textbf{Key Rolling.}\label{ssec:signal-keyrolling}
It is critical that the server maintain a
database of ephemeral identities previously registered on the system
in order to check for re-use of old ephemeral identities.
% If this list is not properly maintained, a user can reuse old 
% ephemeral identities after they have been exchanged for a fresh
% identity in the mailbox closing subroutine.  This would, in effect,
% give users an unlimited number of valid ephemeral identities, 
% undermining the intended abuse prevention goal.
Note that to 
prevent reuse, this database must be maintained for as long as the
identities are valid and grows with the number of mailboxes, not the
number of users.

We suggest that Signal update their mailbox signing key at regular
intervals, perhaps each day, and leave two or three keys valid for overlapping
periods of time to avoid interruptions in service.
Because the validity of a signed
mailbox key is tied to the signing key, each update allows the server to
``forget'' all the keys
that it saw under the old signing keys as they cannot be reused.
% rolling the valid signing
% keys purges the validity of all older, signed ephemeral identities.
% Importantly, this also allows the server to ``forget'' all the keys

% Because Signal is an international service, rolling at a particular
% time each day may cause logistical problems with certain parts of
% the globe.
% For instance, rolling keys during the busiest time of
% the day may cause increased traffic at incontinent times.u
% Therefore,
% we also suggest that signing keys could be valid for overlapping 
% periods of time, giving users flexibility.  The only added
% complication is that a fresh ephemeral identity generated when
% closing a mailbox must be signed with the signing key used to sign
% the original mailbox identity.

\medskip \noindent
\textbf{Mailbox Opening.}
It is important that users perform {\bf Get signed mailbox key} (where Signal
learns a user's identify) and {\bf Open a mailbox} in an uncorrelated way.
Otherwise, Signal could link the two and identify the anonymous mailbox.
We recommend performing {\bf Get signed mailbox key} at regular intervals (e.g.
the same time each day), but careful consideration must be taken for users
that are offline during their usual time. Users should not come online and
perform both operations immediately if sending to a new conversation. To avoid this,
clients should maintain a small batch of extra signed mailbox keys for new
conversations.
%and refresh these at random intervals uncorrelated with their use.


\medskip \noindent
\textbf{Cost Overhead.}
We analyze the worst case cost of scaling our protocol. We generously assume that 10 million anonymous mailboxes will be opened every day.
%, and compute the costs associated with this many messages.
The server's part of opening these mailboxes constitutes calls to 
\bsverify and \bssign and a database query (to check for repeated identities).
In our experiments, the
two blind signature operations, including the Django networking
interface, took a cumulative .014 seconds.  Using AWS Lambda, supporting 
10 million messages each day would cost approximately \$10 per month.  
We estimate that doing 10 million reads and writes a day to a DynamoDB database
would cost approximately \$20 per month, using AWS's reserved capacity pricing.

Using the key rolling scheme described above, the database contains
at most the number of messages delivered in a day times the number
of simultaneously valid keys. At 10 million messages each time with a
two overlapping valid keys, this means the database would contain at most 20 million
ephemeral identities.  Assuming  256-bit identity
values, the entire database would never exceed a few GB of data.
%AWS charges \$0.25 to store a GB of data each month.
Therefore, we conservatively estimate that the
marginal cost of supporting our protocol for 10 million ephemeral identities per day
would be under \$40 per month.  We note our analysis does not consider the personnel cost associated with
developing or maintaining this infrastructure. Ideally, this would be
amortized along with Signal's existing reliability and support infrastructure.
% and given the relatively simple design change is unlikely to create significant
% additional burden.



%%% Local Variables:
%%% mode: latex
%%% TeX-master: "main"
%%% End:
