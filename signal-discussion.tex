\section{Discussion}
\label{sec:signal-discussion}

\subsection{Other solutions}\label{sec:signal-othersolutions} % list of other solutions, why they fail to live up to ours
In this section, we consider alternative, \emph{minor
changes} to the existing sealed sender protocol and evaluate their effectiveness.

% While our solution presented in \cref{sec:solution} is
% designed for practicality, there may be alternative ``lightweight'' ways to
% provide anonymity between users.

\medskip \noindent
\textbf{Random delays.}
Users could send delivery or read receipts after a
random delay, making it harder for attackers to correlate messages.
This forces an attacker to increase the \emph{epoch duration} to
perform the same attack. We analyze the effect of varying epoch duration in
\cref{fig:signal-combinedfigure}, and find that even
with hour-long epochs---likely rendering delivery receipts useless---users could
still be identified within 60 messages.
We conclude that injecting random delays is an ineffective way to achieve anonymity.


\medskip \noindent
\textbf{Cover traffic.}
Users could send random sealed-sender messages that are transparently
ignored by the recipient in order to cover for the true pattern of ongoing
conversations. Based on our experiments, we again see that cover traffic slows
down our attack, but at a linear rate with the amount of extra traffic: even
with 10x extra messages, the anonymity set of potential senders to Bob after 100
messages is under 1000 users. This
mitigation strategy has obvious costs for the service provider, without
significant benefit to user anonymity.

\medskip \noindent
\textbf{Disable automatic receipts.}
While Signal users can disable read receipts and typing notifications,
they currently cannot turn off delivery receipts. Adding an option for this would
give users the choice to greatly mitigate this attack. We note disabling would
have to be mutual: Alice turning off delivery receipts should also prevent Bob
from sending them, different from how Signal currently disables read receipts.
We also note users could potentially still be linked purely by their messages
eventually, making this only a partial mitigation.




%Based on our previous results, we can see that adding extra random cover
%traffic would slow down our attack, but at a rate that scales linearly with
%the amount of extra traffic needed. For instance,
%even with only 10\% of messages generating responses to Alice (i.e. roughly a 10x overhead in
%extra messages), the anonymity set of potential senders to Bob after 100 messages is
%below 1000 users. This mitigation strategy has obvious costs for the service provider,
%which would need to handle a large number of extra messages
%for the same-size user base. 

%\dsr{Important: need to add more concrete numbers here about
%percentages. It would be good to add a statement like, ``from the
%previous section, we showed that the number of messages needed to
%identify Alice scales linearly with the amount of extra random traffic.
%Even if only 10\% of messages are responded to, Alice can still be
%identified after X messages.}


\subsection{Drawbacks and Likelihood of Adoption}
% why won't signal use this, potential steps to help adoption.

We believe that the solution we have proposed in \cref{sec:signal-solution}
is both practical and cost-effective.  However, there are a few drawbacks.
Most importantly, it adds complexity to the system, and complexity always increases the
likelihood of error and vulnerability.  In particular, the key rolling scheme
we suggest in \cref{ssec:signal-keyrolling} requires increased complexity in
the back-end key management system.  While the compromise of these keys would
not leak message content, it could allow for a cheap resource denial attack on
Signal.

A second important drawback of our solution is the assumption that a malicious
service provider cannot use network information to identify users. As mentioned,
using Tor~\cite{tor_two,guardianproject} would address this, but only if enough
users did so to increase the anonymity set.

\if0
A second important drawback of our solution is the simplifying assumption that
a malicious service provider cannot use network information to identify users.
In practice, removing this attack vector would require the use of slow
networking primitives like Tor~\cite{tor_two}.  Routing this much traffic through Tor might
have a impact on the speed of the network, but might also provide additional cover 
traffic to increase the anonymity set of Tor users. Users can also choose to
selectively use Tor on their own (e.g. with Orbot~\cite{guardianproject}, an Android Tor app that
routes the whole phone through Tor).
\fi


Finally, our ephemeral identities may increase complexity for users that use
Signal on multiple devices. Signal would need to securely share or deterministically
generate these keys with other devices in a privacy-preserving way.

\if0
Finally, using ephemeral identities could cause synchronization issues for users
that use Signal on multiple devices.  We have suggested that the keypairs associated
with the identities should be generated locally by a single device.  This key 
pair would have to be rapidly and securely shared among multiple devices. 
\fi

Given the limited scope and impact of these drawbacks, we believe that is
reasonable to believe that Signal or other secure messengers could potentially
adopt our solution. %\gsk{do we need more here?}


\subsection{Group messaging}
\label{sec:signal-group}

%\dsr{Previous subsection title was ``Multiple messages'', but I wasn't
%sure what that was about so I wrote about group messaging instead. If
%``Multiple messages'' was supposed to mean something else, please add
%it!}

The OTR and Signal protocols were first designed for pairwise
communication, and we have focused on such conversations in this work.
However, group messaging is an important use case for private messaging
services, and has recently shown to be vulnerable to different kinds of
attacks~\cite{RMS18,CCGMM18,SH19}.

An interesting direction for future work would be to extend our attacks
to this setting. It is clear that received receipts and read receipts
do not work the same way in groups as they do for two-way conversations.
On the other hand, group messages have additional \emph{group management
messages} which are automatically triggered, for example, when a new
member attempts to join the group. It would be interesting to understand
if, for example, our attack could exploit these message to de-anonymize
\emph{all} members of a given group chat.

Fortunately, it does seem that our main solution proposed in
\cref{sec:signal-solution} would be applicable to the group chat setting: all
members of the group chat would create new, anonymous mailboxes used
only for that particular group. However, this would still leave the
difficulty of the initial configuration and key management, which would
be more complicated than that two-party setting. We consider this to be
important and useful potential future work.
