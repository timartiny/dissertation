\abstract{  \OnePageChapter	% because it is very short
	The vast majority of online services are run using a centralized
	infrastructure. The centralized nature of these services allow the provider
	to have absolute control over the content as well as any profit generated by
	that content. Centralized services often have servers distributed across the
	world for user reliability, though they function as centralized systems:
	their functionality is identical across their network and the data they
	collect is available to the service as a whole, not only the server a user
	interacts with. Users of these services, and the data they generate, are
	completely at the whim of the provider; companies often offer vague promises
	of security and privacy of the data collected from their users which the
	users cannot verify themselves. Moving these services to a decentralized
	system (where each server acts independently of others) could address these
	issues but decentralized systems often face severe scalability issues as
	well as having cumbersome requirements on users such as needing specialized
	software to access the service.

    This dissertation demonstrates that user privacy can be inherently built
    into centralized systems using cryptographic protocols. Centralized systems
    can offer their services with minimal user information allowing services
    \emph{actually} geared towards privacy to have it be a core functionality of
    their service. A service that doesn't have access to users data cannot abuse
    or leak it.

    \emph{Proof of Censorship} utilizes Private Information Retrieval to allow
    content providers (such as Twitter) to be cryptographically auditable over
    whether content has been modified or removed. In 2018 Signal (a secure
    end-to-end messenger) introduced Sealed Sender in an attempt to hide the
    sender of encrypted messages. \emph{Improving Signal's Sealed Sender}
    strengths Sealed Sender by guaranteeing that privacy cryptographically using
    blind RSA signatures. \emph{Mind the IP Gap} measures how countries utilize
    their centralized censorship apparatus to restrict content to users through
    DNS manipulation. 

	}