\medskip \noindent
\textbf{Blind Signatures}
The mechanism to prevent abuse for the creation of anonymous accounts relies on the 
cryptographic primitive of \emph{blind signatures}, as first proposed by
\cite{C:Chaum82}. Blind signature schemes
have 5 algorithms: \bskeygen, \bsblind, \bssign, \bsextract and \bsverify.  
\bsblind takes in the public key of the signer, a message, and some randomness and
outputs a blinded message.% and a blinding factor.
\ \bssign takes in
the signer's private key and a blinded message and outputs a 
blinded signature.  \bsextract takes in
%the blinding factor produced by \bsblind and
a blinded signature and the randomness used in blinding and outputs a normal 
signature.  Finally, \bsverify takes in a message and the 
signer's public key and decides if the signature is valid.

% following steps between a client (message generator) and server
% (signer).

The interaction between a server with the signing keypair $sk, pk$ and a client is as follows:
\begin{enumerate}[noitemsep]
  \item Client generates the blinded message % and blinding factor $(b, bf)
  \\ $b \gets \bsblind(m, pk;r)$ for $r \sample \bin^\securityparameter$
  \item Client sends $b$ to the server for signing.
  \item Server computes the blinded signature 
  \\ $s_{blind} \gets \bssign(b,sk)$ and returns it to the client.
  \item Client extracts the real signature % $s = \bsextract(s_{blind}, bf, pk)$
  \\ $s \gets \bsextract(s_{blind}, pk; r)$
  \item Client, in a different network connection, sends the initial message $m$ and the real signature $s$ to the server, who runs $\bsverify(pk,m,s)$ 
\end{enumerate}
The blind signature scheme should have the usual signature
unforgeability property.  Additionally, it should be impossible for a
server to link the blinded message and blinded signature to the real
message and real signature.  We use the RSA-based construction of blind
signatures from \cite{C:Chaum82}.
%%% Local Variables:
%%% mode: latex
%%% TeX-master: "main"
%%% End:
